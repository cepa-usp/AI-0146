\documentclass{article}
	\usepackage[utf8]{inputenc}
	\usepackage[T1]{fontenc}
	\usepackage{bookman}
	\usepackage[brazil]{babel}
	\usepackage{amssymb}
	\usepackage{indentfirst}

	\newcommand\foreign[1]{\textsl{#1}}
	\hyphenation{u-su-á-ri-os}
\begin{document}

\section*{Limites}

O objetivo deste objeto de aprendizagem é facilitar a compreensão da definição formal de limite de uma função de uma variável, $f : \mathbb{R} \to \mathbb{R}$. Para isso o usuário conta com um eixo $x$ (domínio de $f$) e um eixo $y$ (imagem de $f$).

No eixo $x$ estão definidos o ponto $x_0$, onde se quer calcular o limite, e $x$, um ponto na vizinhança de $x_0$, tal $|x_0 - x| < \delta$. Esta condição é representada na figura pela impossibilidade de arrastar $x$ para longe dos colchetes ao redor de $x_0$, que indicam os limites inferior e superior da vizinhança.

No eixo $y$ estão definidos $f(x)$, que só pode ser alterado arrastando-se $x$, bem como $L$, o limite procurado, e a vizinhança de $L$, de tamanho $\varepsilon$.

A função $f$ é escolhida aleatoriamente pelo \foreign{software} e sua expressão \emph{não} é informada. Ainda assim, o usuário pode determinar o limite de $f$ em $x_0$ modificando $\delta$ e $L$ de tal modo que o ponto $f(x)$ fique sempre na vizinhança de $L$.

A atividade fica muito mais interessante se explorada com dois usuários: o primeiro define $x_0$ e $\varepsilon$, enquanto o segundo comanda $x$, $\delta$ e $L$. Assim, o primeiro usuário pede ao segundo que ajuste $L$ e $\delta$ de modo que, para qualquer $x$, $f(x)$ fique sempre na vizinhança de $L$ (matematicamente, $|L - f(x)| < \varepsilon$). Quando ele conseguir fazer isso, o primeiro usuário reduz $\varepsilon$ e novamente solicita ao primeiro que ajuste seus parâmetros. Esse jogo de gato-e-rato pode prosseguir indefinidamente, até que ambos os usuários concordem que $L$ já representa, com adequada acurácia, o limite procurado.

Eventualmente o \foreign{software} sorteará uma função descontínua, o que permite ainda explorar a inexistência do limite: o usuário não conseguirá encontrar um $L$ e $\delta$ que satisfaça a condição acima. Neste ponto é possível explorar também o conceito de limite lateral.

Espera-se que, procedendo desta forma, torne-se mais fácil para o usuário compreender a definição formal de limite: o limite de uma função $f : \mathbb{R} \to \mathbb{R}$ em $x_0$ é igual a $L$ se, para qualquer $\varepsilon > 0$ dado, for possível encontrar um $\delta > 0$ tal que $|L - f(x)| < \varepsilon$ para qualquer $x$ que satisfaça $|x_0 - x| < \delta$.


\end{document}
