\documentclass[fleqn,12pt]{article}
    \usepackage[utf8]{inputenc}
    \usepackage[T1]{fontenc}
    \usepackage{bookman}
    \usepackage[brazil]{babel}
    \usepackage{xcolor}
    \usepackage{amsmath}
    \usepackage{amsfonts}
    \usepackage[margin=2cm]{geometry}
    \usepackage{txfonts}
    \usepackage{xspace}
    \usepackage{url}
    \usepackage{indentfirst}
    
    
    % Minhas definições
    \newenvironment{ct}{\begin{quotation}\color{red!30!black}\sffamily\small}{\end{quotation}} % Comentários técnicos
    \newenvironment{cp}{\begin{quotation}\color{green!30!black}\sffamily\small}{\end{quotation}} % Comentários didático-pedagógicos
    \newcommand\ctc[1]{\textcolor{red!30!black}{{\sffamily#1}}} % Comentário técnico curto
    \newcommand\cpc[1]{\textcolor{green!30!black}{{\sffamily#1}}} % Comentário didático-pedagígico curto
    \newcommand\foreign[1]{\textsl{#1}}
    \newcommand\etc{\foreign{etc}}
    \newcommand\proceed{\textcolor{green}{$\medbullet$}\xspace}
%    \newcommand\detect{\textcolor{red}{$\medbullet$}\xspace}
 %   \newcommand\pt{\ensuremath{(x,y)}\xspace}
    \newcommand\eg{\foreign{eg}}
 %   \newcommand\code[1]{\textsf{#1}}
    \newcommand\ie{\foreign{ie}}
    \newcommand\condicional[2]{$\lfloor$%
	\textsf{\textcolor{blue}{{\footnotesize #1}}}
	$\leadsto$ #2%
	$\rfloor$}
  %  \newcommand\ptzero{\ensuremath{(x_0,y_0)}\xspace}
    \newcommand\forum[1]{\colorbox{yellow}{Discutir no fórum: #1}} 

    % Título e autor
    \title{Roteiro de produção da AI-0146 (limites)}
    \author{Dr. Ivan Ramos Pagnossin}
   \date{\today}

%    \hyphenation{vi-zin-han-ça va-ri-á-ve-is}

\begin{document}

    \maketitle
    


    %-------------------------------
    \section{Limites e continuidade}

    \begin{ct}
	Para este passo-a-passo a AI-0146 deve ser configurada de modo que $f(x) = 2x + 1$.
    \end{ct}

    Nesta atividade interativa veremos, passo-a-passo, o conceito de limite de uma função de uma variável.

    O valor $f(x)$ para $x = x_0$ é simplesmente $f(x_0)$. Mas e se $x$ for um número muito \emph{próximo} de $x_0$, embora \emph{diferente}? Intuitivamente, podemos responder: ``$f(x)$ será muito próximo de $f(x_0)$'' (é possível finalizar esta frase com ``embora diferente''?). Este realmente é o caso \emph{na maioria das vezes}, mas nem sempre. Veja esta equação:
    \begin{equation*}
	\lim_{\scriptsize x  \to x_0} f(x) = L.
    \end{equation*}

    Ela deve ser lida assim: o limite de $f$ quando $x$ tende a $x_0$ ($x \to x_0$) é igual a $L$. Ou seja, se $x$ for muito \emph{próximo} de $x_0$, $f(x)$ será \emph{igual} a $L$, um número que não conhecemos a priori. $L$ é dito ``o limite'' de $f$ em $x_0$.

    Dois detalhes podem ter passado despercebidos: primeiro, $L$ não é necessariamente igual a $f(x_0)$ (veremos que isto tem a ver com a \emph{continuidade} de $f$). Segundo, o que significa dizer que $x$ é ``próximo'' de $x_0$?

    Para responder a esta pergunta, veja a figura acima \ctc{(AI-0146)}. $x_0$ é um \emph{ponto interior} ao domínio de $f$ (você pode arrastá-lo), enquanto $x$ é um ponto qualquer da vizinhança de $x_0$ (experimente arrastá-lo também). Dizer que $x$ é ``próximo'' de $x_0$ significa dizer simplesmente isso: que $x$ está na vizinhança de $x_0$, ou ainda que $|x - x_0| < \delta$, onde $\delta$ é a ``amplitude'' da vizinhança (veja a figura).\proceed

    \forum{Por que $x_0$ deve ser um ponto interior ao domínio de $f$?}

    A função $f$ é calculada sempre em $x$ (nunca em $x_0$) e resulta no número $f(x)$, também apresentado na figura, na reta $\mathbb{R}$ à direita [você só pode alterar $f(x)$ mexendo em $x$]. Há também o ponto $L$, no centro do intervalo aberto $]L - \varepsilon, L + \varepsilon[$, com $\varepsilon > 0$. Arraste $L$ e os colchetes ao redor dele para entender a construção da figura antes de continuar. 

    Caso não se lembre, dizer que $f(x)$ pertence ao intervalo $]L - \varepsilon, L + \varepsilon[$ é o mesmo que escrever:
    \begin{equation*}
	L - \varepsilon < f(x) < L + \varepsilon.
    \end{equation*}

    Ou ainda, que a distância de $f(x)$ até $L$ é menor que $\varepsilon$:
    \begin{equation*}
	|L - f(x)| < \varepsilon.
    \end{equation*} \proceed

    \forum{Por que o \foreign{software} não permite mover $f(x)$? Qual é a implicação disso em $x$?}

    Com isto tudo em mente, dizer que $L$ é o limite de $f$ quando $x$ tende a $x_0$ significa o seguinte: se \textbf{eu} escolher um $\varepsilon$ arbitrariamente pequeno, \textbf{você} pode encontrar um $\delta$ tal que, para qualquer $x$ na vizinhança de $x_0$, $f(x)$ pertence ao intervalo $]L - \varepsilon, L + \varepsilon[$. \proceed

    Na figura acima você pode escolher $L$ ao bel prazer (arraste-o), mas apenas para um valor será possível satisfazer o critério do parágrafo anterior. Este é o limite procurado.

    \begin{ct}
	Na AI-0146, faça $x_0 = 2$ e $\varepsilon = 3$, fixando-os de modo a impedir o usuário de alterá-los a partir daqui.	
    \end{ct}

    Parece complicado, mas a ideia é simples. Vamos experimentá-la passo-a-passo na figura acima. Nosso objetivo é determinar $L$, partindo de um valor arbitrariamente escolhido (escolhi $x_0 = 2$). Não é necessário preocupar-se com valores numéricos, embora você possa vê-los passando o \foreign{mouse} por cima de $L$, $\varepsilon$, \etc.

    \paragraph{Passo 1} Coloque $x_0$ próximo de $2$ e ajuste $\delta \approx 1$ (de modo que a vizinhança fique toda dentro do domínio de $f$). Este é o ponto no qual calcularemos o limite. \proceed

    \begin{ct}
	Faça $x_0 = 2$ e $\delta = 1$ na AI-0146 se os valores ajustados pelo usuário diferirem de mais de $10\%$ desses valores.

	Fixe $x_0$ e $\delta$ na AI-0146, impedindo o usuário de alterá-los a partir daqui.
    \end{ct}

    \paragraph{Passo 2} Arraste $L$ para a direita, até $20$, mais ou menos. \proceed

    \begin{ct}
	Faça $L = 20$ na AI-0146 caso o valor ajustado pelo usuário difira de mais de $10\%$ desse valor.

	Fixe $L$ na AI-0146, impedindo o usuário de alterá-lo a partir daqui.
    \end{ct}

    Pela definição acima, se $L \approx 20$ for o limite procurado, então podemos arrastar $x$ livremente e sempre obteremos $f(x)$ entre os colchetes ao redor de $L$. Veja se isto acontece. \proceed

    \begin{ct}
	Libere $L$ na AI-0146, permitindo ao usuário alterá-lo a partir daqui.
    \end{ct}

    Como você pode perceber, em nenhum momento $f(x) \in ]L - \varepsilon, L + \varepsilon[$. Logo, $L \approx 20$ não é o valor que procuramos. Na verdade, $L$ deve estar mais próximo de $f(x)$.

    \paragraph{Passo 3} Escolha um $L$ mais próximo de $f(x)$. Arraste $x$ por todo o intervalo $]x_0 - \delta, x_0 + \delta[$ e observe o movimento de $f(x)$ para tomar a sua decisão.

    \begin{ct}
	Se o usu
    \end{ct}

    \condicional{Usuário escolheu $3 \le L \le 7$}{Muito bom! Parece que você entendeu a ideia.} Eu ajustei $L = 6$ apenas para simplificar, mas poderíamos prosseguir com o valor que você escolheu. Qual é o comportamento de $f(x)$ agora? \proceed

    Desta vez $f(x)$ hora está dentro do intervalo, hora não. O problema é que a vizinhança de $x_0$ está muito grande.

    \paragraph{Passo 4} Faça $\delta \approx 0,2$. \proceed

    \begin{ct}
	Faça $\delta = 0,2$ na AI-0146 independentemente do ajuste do usuário.

	Fixe $\delta$, impedindo o usuário de alterá-lo a partir daqui.
    \end{ct}

    Agora $f(x)$ está totalmente contido no intervalo $]L - \varepsilon, L + \varepsilon[$. \proceed No entanto, ainda pela definição de limite, \textbf{eu} posso reduzir $\varepsilon$. Escolhi $\varepsilon = 1$, diminuindo assim a extensão desse intervalo, e novamente colocando $f(x)$ fora dele. Verifique. \proceed

    \begin{ct}
	No primeiro \proceed do parágrafo acima, faça $\varepsilon = 1$ na AI-0146 (continua fixo, não suscetível à ação do usuário).
    \end{ct}

    O que devemos fazer para corrigir isto é, novamente, aproximar $L$ de $f(x)$.

    \paragraph{Passo 5} Faça $L = 5$ e verifique o comportamento de $f(x)$. \proceed

    \begin{ct}
	Faça $\varepsilon = 0,5$ na AI-0146 (continua fixo, não suscetível à ação do usuário).
    \end{ct}

    Parece que $f(x)$ voltou a manter-se entre os colchetes. Mas agora eu reduzi ainda mais $\varepsilon$, para $0,5$.

    \paragraph{Passo 6} Reduza $\delta$ mais um pouco (não se esqueça da ferramenta de \foreign{zoom}), de modo a devolver $f(x)$ para o intervalo $]L - \varepsilon, L + \varepsilon[$.

    \begin{ct}
	Se o usuário não tiver reduzido $\delta$, faça $\delta = 0,05$ na AI-0146.
    \end{ct}

    Você percebe que eu posso continuar reduzindo $\varepsilon$ e você, continuar ajustando $L$ e $\delta$ de modo a manter $f(x)$ no intervalo $]L - \varepsilon, L + \varepsilon[$ para qualquer $x$ na vizinhança de $x_0$?

    \paragraph{Passo 7} Pressione o botão [ÍCONE ZOOM MENOS] algumas vezes para voltar à visualização inicial e responda: quanto vale $L$?

    $L = [\qquad]$ \proceed

    \begin{ct}
	Disponibilize um \foreign{text field} para o usuário digitar a resposta. O valor esperado é $L = 5$, com erro aceitável de 10\%.
    \end{ct}

    \condicional{Usuário errou}{A resposta correta é $L = 5$. Se você não chegou a este resultado (aproximadamente), refaça os passos acima antes de prosseguir.} Matemanticamente, escrevemos

    \begin{equation*}
	\lim_{x  \to 2} f(x) = 5.
    \end{equation*}

    Finalmente, note que $L = f(2)$. Em palavras, o limite de $f$ em $x_0$ é simplesmente $f(x_0)$. Quando isto acontece, dizemos que \emph{$f$ é contínua em $x_0$}. E quando isto acontece para \emph{todos} os pontos do domínio de $f$, dizemos simplesmente que \emph{$f$ é contínua} (em todos os pontos). \proceed

    Acontece que este nem sempre é o caso. A figura acima agora ilustra uma outra função, e que não é contínua. Refaça os passos anteriores na busca por $L$ para $x_0 = 2$ e discuta com seus colegas: é possível determinar o limite dessa função em $x_0 = 2$? Por que? E que função é essa? [DESENVOLVER PASSOS PARA ESTE CASO TAMBÉM].

\end{document}
